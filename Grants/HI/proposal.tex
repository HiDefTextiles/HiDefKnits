\documentclass{article}
\usepackage{geometry}
\geometry{a4paper, top=2.5cm, bottom=2.5cm, left=2.5cm, right=2.5cm}

\usepackage{changepage}
\usepackage{etoolbox}
\usepackage{setspace}
\onehalfspacing

\title{HiDef Textiles: Optimizing Textile Processes with AI}
\author{Dr. Helga Ingimundardóttir}
\date{\today}

\usepackage{natbib}
\bibliographystyle{plainnat}

\usepackage[inline]{enumitem}
\setlist[enumerate,1]{
    label=\textit{\roman*)},
    before=\unskip{: },
    itemjoin={{; }}, itemjoin*={{, and }},
    after={{. }}
}

\BeforeBeginEnvironment{abstract}{\begin{adjustwidth}{1cm}{1cm}\small\singlespacing}
\AfterEndEnvironment{abstract}{\end{adjustwidth}}

\begin{document}
    \maketitle

    \begin{abstract}
        \textbf{HiDef Textiles} uniquely merges AI, industrial engineering, computer science, and textile craftsmanship,
        enhancing embroidery, sewing, knitwear design and more. Our vision is to democratize tech and programming via
        textiles, seamlessly integrating arts and sciences in an engaging manner.

        We are building a comprehensive learning platform, emphasizing sustainable practices and efficiency, and are
        committed to sharing knowledge and industrial best practices across all skill levels. Our educational
        integration fosters a collaborative space where arts, mathematics, and engineering converge.

        With a strong focus on encouraging women in STEM, we present textiles as a tangible entry point to
        engineering and computational concepts, aiming to cultivate an inclusive community and drive innovation.

        In seeking funding, our objective is to strengthen our STEAM approach, ensuring a rich, interdisciplinary
        experience that merges STEM with the Arts. This financial support will enable us to employ industrial
        engineering and computer science students for the development of both the content and the platform itself,
        thereby fostering a culture of knowledge and innovation.

        \noindent \textbf{Keywords:} \textit{Process Optimization}, \textit{Sustainable Textile Manufacturing},
        \textit{Interdisciplinary Innovation} and \textit{Generative Design}.
    \end{abstract}


    \section{Introduction}

    Textiles, historically tied to women's work, blend skill, precision, and creativity. Ada Lovelace
    saw in the Jacquard loom an algorithmic language, linking textile arts to computation. As
    \citet{sadieplant} underscores, this connection highlights textiles' potential to inspire women in
    STEM, offering a tangible way to grasp engineering and computational concepts.

    HiDef Textiles embodies a collaborative effort to transform textile craftsmanship, merging
    traditional arts with modern technology and interdisciplinary innovation. By integrating industrial
    engineering, computer science, and creative design, it forges a bridge from heritage crafts to
    contemporary practices. This initiative not only enhances innovation but also champions diversity,
    with a particular focus on inviting women into the field. Research by \citet{qiu2013,keune2022}
    attests to the unique appeal of computational textiles, demonstrating their potential to demystify
    technology and programming. In doing so, HiDef Textiles weaves together artistic expression and
    engineering precision, making industrial processes more accessible and inviting.


    \section{Projects}
    HiDef Textiles merges the art of traditional textiles with the efficiency of modern technology and
    computational methods. We are working hard to improve textile craftsmanship, aiming for greater efficiency,
    accessibility, and sustainability in the field. Our key projects include
    \begin{enumerate*}
        \item Optimization in Embroidery: We're fine-tuning machine embroidery processes to increase precision
        and minimize waste
        \item Fabric Utilization in Sewing: Our goal is to optimize fabric use, promoting sustainable practices
        in the process
        \item AI-Enhanced Knitwear Design and Visualization: We're using advanced technology to make knitwear
        design more accessible to everyone
    \end{enumerate*}
    Through these initiatives, we're working toward a future that honors the rich heritage of textiles while welcoming
    modern innovations.

    \subsection{Optimization in Embroidery}

    Machine embroidery requires precision, and our initiative is aimed at boosting its
    efficiency. We are streamlining the workflow from design to stitching, employing
    operations research optimization techniques. Our goal is to significantly reduce
    setup and cleanup times, a major bottleneck in the embroidery process.

    In partnership with \citet{inkstitch}, an open-source embroidery design tool, we are
    working to integrate shortest route algorithms to enhance their toolkit. This task
    involves optimizing stitch points for efficiency and aesthetic appeal. Our efforts are
    streamlined to save time in both industrial setups and for hobbyists, ensuring
    precision and pleasing visual outcomes.

    \begin{description}
        \item[Objective:] Integrate real-world application of shortest path problems in embroidery to enhance
        learning in \emph{Operations Research}.
        \item[Milestone 1:] Introduce basic concepts of shortest path problems to students.
        Students implement basic version for hands-on experience.
        \item[Milestone 2:] Add an additional layer of complexity to the problem by introducing an optimization for
        the start/end nodes in the sub-paths.
        \item[Milestone 3:] Develop advanced scenarios for master's projects focusing on processing paths for satin
        stitches, how they should meet and/or cross-over, and how to optimize the path for aesthetic appeal.
    \end{description}

    \subsection{Fabric Utilization in Sewing}

    Sewing requires precise pattern placement and fabric cutting, particularly in garment creation.
    This initiative aims to optimize these aspects using advanced algorithms. We address challenges
    like irregular fabric shapes from previous projects and unique fabric types. Our goal is to
    minimize fabric waste, providing adaptable solutions for various fabric widths and suggesting
    ideal orientations for pattern pieces.

    We draw on methods from the 2D Free Form Bin Packing and 2D Irregular Cutting Stock Problems
    \citep{bennell2009tutorial, xu2016efficient}, targeting home sewists and small-scale garment
    makers. Our project helps these groups maximize fabric use, conserve resources, and adopt
    sustainable sewing practices. These users often work with non-uniform fabric shapes on a
    small scale, lacking access to advanced tools. Our open-source approach is thus invaluable.

    \begin{description}
        \item[Objective:] Progressively deepen students' proficiency in Fabric Utilization,
        initiating with straightforward scenarios and gradually introducing more intricate challenges,
        enhancing their ability to efficiently utilize fabric in various contexts.
        
        \item[Milestone 1:] Begin with fundamental principles of fabric utilization, focusing on the
        optimal arrangement of regular shapes (rectangles) from a rectangular piece of fabric, i.e. cutting stock
        problem.
        
        \item[Milestone 2:] Transition to scenarios that involve regular shapes being cut from irregular
        pieces of fabric, teaching students how to maximize fabric usage with remnants or animal hides.
        
        \item[Milestone 3:] Introduce the complexity of cutting irregular shapes from regular fabric
        pieces, emphasizing efficient orientation and placement strategies. A common problem in industrial sewing.
        
        \item[Milestone 4:] Expand to real-world bespoke garment-making challenges, dealing with irregular
        shapes being cut from irregular fabric pieces.
        
        \item[Milestone 5:] Broaden the context to a 3D space, applying fabric utilization principles to
        3D printing, where students learn to place objects on a printer bed, navigating the added
        dimensions and complexities.
    \end{description}


    \subsection{AI-Enhanced Knitwear Design and Visualization}

    Knitting patterns usually specify a single yarn type and a default gauge, but knitters vary in
    their tension, and yarn substitutions are common. Our AI-Enhanced Knitwear Design and
    Visualization project simplifies the adjustment of patterns to individual needs, aiding both
    knitters and designers.

    For knitters, it offers an easy way to adapt existing patterns to their gauge and chosen yarn,
    ensuring a right fit. Designers benefit by efficiently creating versatile patterns that
    consider varying gauges and yarn types right from the start.

    \begin{description}
        \item[Objective:] Leverage extensive databases like Ravelry\footnote{Ravelry is an extensive online
        community and database dedicated to the arts of knitting, crocheting, spinning, and weaving. Founded in
        \citeyear{Ravelry} by \citeauthor{Ravelry}, it provides a comprehensive platform for fiber artists to
        share patterns, discuss techniques, and showcase finished projects. Users can browse and purchase
        patterns, track their own projects, and connect with other enthusiasts across the globe. The platform
        hosts millions of user-generated patterns, making it an invaluable resource for understanding current
        trends, popular designs, and common modifications made to existing patterns. For our project, Ravelry
        serves as a crucial data source, offering a rich dataset that can be analyzed to gain insights into user
        preferences and behavior, and to train our generative AI models for pattern design and visualization.}
        to understand trends, popular patterns, and scope of pattern modifications. Develop generative AI
        applications for pattern design.
        \item[Milestone 1:] Analyze Ravelry dataset (meta-data) for trends and pattern popularity.
        \item[Milestone 2:] Develop generative AI for Fairisle patterns, using images of lopapeysa for training.
        \item[Milestone 3:] Identify and work on public domain patterns for future generative AI applications.
        \item[Milestone 4:] For complex knitwear like lace techniques, solicit participation from knitwear designers
        for pattern instruction sets.
        \item[Milestone 5:] Optimize patterns based on perceived quality, and reproduce using AI.
    \end{description}


    \section{Academic and Societal Value}

    HiDef Textiles integrates innovation and education to generate societal impact.
    Our initiatives span from major industrial applications to personal projects, ensuring wide-reaching relevance.

    \subsection{Educational Integration and Student Participation}

    Incorporating our projects into industrial engineering courses,
    such as Operations Research and Business Intelligence, facilitates unique learning experiences.
    We actively encourage students to engage through smaller projects, enhancing their understanding of the
    practical applications of their studies. The goal is to also attract a master's student, potentially funded,
    to contribute significantly to one of our key projects.

    \subsection{Community Empowerment and Tool Accessibility}

    Our open-source tools serve the community, empowering users to take charge of their textile projects
    while understanding the processes behind them.
    The HiDef Textiles web platform is pivotal in this, providing a space for sharing, collaboration,
    and continuous learning.

    \subsection{Academic Contributions and Simplified Learning Resources}

    While we focus on producing academic findings to contribute to scholarly journals, we equally
    prioritize simplifying these findings. The aim is to transform them into educational resources,
    making complex concepts accessible and understandable, thereby broadening their impact and applicability.

    This comprehensive approach ensures that HiDef Textiles not only contributes to the academic world
    but also fosters a community of informed users and enthusiastic learners,  bridging the gap between theory and
    practice in textile arts.


    \section{Conclusion}

    HiDef Textiles redefines textile manufacturing, balancing efficiency, accessibility, and sustainability
    while honoring traditional craftsmanship.
    We offer innovative learning avenues, promoting academic and societal progress.

    Integrating STEM into textile arts, we aim to attract a diverse audience,
    particularly encouraging women to explore optimization and operations research within industrial engineering.
    Our open-source platform fosters community, innovation, and a sustainable future in textiles and beyond.

    \newpage
    \bibliography{references}


\end{document}