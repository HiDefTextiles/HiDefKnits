\documentclass{article}
\usepackage{geometry}
\usepackage{lipsum} % For generating dummy text
\geometry{a4paper, margin=1in} % Adjust margins as needed

\title{HiDef Textiles: AI-Driven Process Optimization and Generative Design in Textiles Manufacturing}
\author{Dr. Helga Ingimundardóttir}
\date{\today}

\usepackage[style=authoryear,backend=biber]{biblatex}
\addbibresource{references.bib}

\usepackage[inline]{enumitem}
\setlist[enumerate,1]{
    label=\textit{\roman*)},
    before=\unskip{: },
    itemjoin={{; }}, itemjoin*={{, and }},
    after={{. }}
}
\begin{document}
    \maketitle

    \abstract{
        \textbf{HiDef Textiles} is an AI-driven textile design
        initiative that strives to enhance and refine the practices of textile craftsmanship, seamlessly integrating
        industrial engineering, computer science, and traditional arts.
        Our objective is to optimize diverse textile techniques, fostering the development of accessible and user-friendly
        tools for embroidery, sewing, and bespoke knitwear design.
        Our open-source initiatives prioritize sustainability, waste minimization, and efficiency, making industrial best
        practices readily available to hobbyists and professionals.
        Integrating these projects into educational curricula fosters a unique learning environment, bridging arts,
        mathematics, and engineering. This proposal seeks funding to support these transformative ventures, contributing to
        societal enrichment through knowledge and innovation.

        \textbf{Keywords:} \textit{Process Optimization}, \textit{Sustainable Textile Manufacturing},
        \textit{Interdisciplinary Innovation} and \textit{Generative Design}.
    }


    \section{Introduction}

    HiDef Textiles represents a collaborative effort to redefine textile craftsmanship through precision, innovation,
    and interdisciplinary integration. It brings together industrial engineering, computer science, traditional
    textile arts, and creative design, bridging the heritage of textile crafts with modern technology.

    Textiles, historically linked to women's work, embody a blend of skill, precision, and creativity. Ada Lovelace
    drew inspiration from the Jacquard loom, perceiving its punched cards as an algorithmic language that connected
    textile arts with computation
    This connection, as highlighted by \cite{sadieplant}, underscores textiles' potential to inspire women in STEM by
    providing a tangible way to grasp engineering and computational concepts.

    Building on these historical ties, HiDef Textiles creates a welcoming space for innovation and diversity.
    Research by \cite{qiu2013} and \cite{keune2022} demonstrates the appeal of computational
    textiles, making technology and programming more accessible. By merging textile artistry with industrial
    engineering, HiDef Textiles makes the field inviting, especially for women.

    Moreover, HiDef Textiles integrates STEM principles directly into textile-making, blending creative expression
    with technical precision. This hands-on approach shows the relevance of STEM concepts in familiar artistic
    contexts, offering new opportunities for participation in complex industrial processes. It fosters confidence and
    a sense of belonging, contributing to a more diverse field of industrial engineering.


    \section{Types of Projects Under Consideration}

    HiDef Textiles stands at the forefront of innovation, seamlessly blending traditional textile arts with modern
    technology and computational methods. Through our initiatives, we strive to enhance textile craftsmanship, improve
    efficiency, and make the field more accessible and sustainable. Our key projects include
    \begin{enumerate*}
        \item Optimization in Embroidery, which focuses on streamlining machine embroidery processes for enhanced precision
        and reduced waste;
        \item Fabric Utilization in Sewing, aiming to optimize fabric use and promote sustainable practices;
        \item AI-Enhanced Knitwear Design and Visualization, an innovative approach to democratize knitwear design
        through advanced technology.
    \end{enumerate*}
    Each of these initiatives represents a step toward a future where the rich heritage of textiles is preserved
    while embracing the possibilities of modern innovation.


    \section{Optimization in Embroidery}

    Machine embroidery involves intricate processes requiring precision. The \emph{Optimization in Embroidery}
    initiative aims to enhance the efficiency of this craft by streamlining the entire workflow, from design input
    to stitching.
    We employ classical operations research optimization techniques to minimize wastage, reduce setup and cleanup times,
    and ensure optimal resource usage, promoting sustainability in embroidery.

    We collaborate with \cite{inkstitch}, an open-source machine embroidery design extension for Inkscape.
    Our focus is on implementing shortest route algorithms, enhancing their toolkit. Identifying nodes for
    aesthetically pleasing outcomes in embroidery poses a challenge, but this integration aims to optimize the
    process, delivering efficient and visually appealing designs. Our endeavor introduces precision and efficiency to
    machine embroidery, benefiting large-scale industrial setups and hobbyists alike.

    \subsection{Fabric Utilization in Sewing}

    Sewing, especially in garment creation, demands precise pattern placement and fabric cutting. The \emph{Fabric
    Utilization in Sewing} initiative focuses on optimizing this process with advanced algorithms, addressing
    challenges such as irregular fabric shapes resulting from previous projects or unique fabric choices. We aim to
    significantly reduce fabric wastage, offering solutions adaptable to varying fabric widths and suggesting optimal
    orientations for pattern pieces.

    To tackle this problem, we draw inspiration from methodologies used in the 2D Free Form Bin Packing problem and the
    2D Irregular Cutting Stock Problem \cite{bennell2009tutorial, xu2016efficient}.
    This project holds particular relevance for home sewists and small-scale garment makers, aiding them in
    maximizing fabric utility, conserving resources, and promoting sustainable sewing practices. Unlike industrial
    setups, these users work on a smaller scale, with limited resources and equipment. Their non-uniform fabric
    shapes make the problem more challenging and require a more nuanced approach. Moreover, they are less likely to
    have access to advanced tools and software, making our project's open-source nature all the more valuable.

    \subsection{AI-Enhanced Knitwear Design and Visualization}

    Knitwear design is a craft rich in tradition, often requiring extensive knowledge and experience to achieve
    intricate patterns and precise fitting. The AI-Enhanced Knitwear Design and Visualization project aims to
    democratize this process, bringing the art of knitwear design to a broader audience. By integrating generative AI
    and 3D modeling with traditional knitting techniques, we provide a platform that facilitates the creation of
    customized garments, tailored to individual preferences and fit requirements. Furthermore, this project offers a
    standardized and intuitive open-source platform for knitting design, featuring advanced visualization and
    prototyping capabilities. Users can visualize their designs in real-time, make adjustments, and experiment with
    different patterns and stitches, all before the actual knitting begins. This ensures that even those new to knitwear
    design can achieve professional-level results, blending the artistry of traditional knitting with the precision and
    innovation of modern technology.


    \section{Academic and Societal Value}

    These projects aim to shine a light on the omnipresence of optimization in daily life, showcasing its application in
    the arts and crafts practiced in homes worldwide. By demystifying complex processes and offering user-friendly tools,
    we hope to inspire a new generation of fiber artists who appreciate the optimization potential in their field, and
    for a new generation of industrial engineers who appreciate their field's potential to improve the lives of people
    in more ways than they can imagine.

    The integration of these projects into educational curricula, in courses such as \textbf{Operations Research} and
    \textbf{Business Intelligence}, ensures that students gain practical experience, applying optimization techniques
    in diverse settings.
    We aim to attract master's students to further develop these topics, contributing to both the academic field and
    practical applications in textiles.


    \section{Web Platform and Community Engagement}


    As the HiDef Textiles project matures, our next step is deploying a comprehensive web platform. This platform
    will host the beta version of our tools, offering direct user access and fostering collaboration and education.
    After development and testing, we'll launch the beta version on a robust web server to gather user feedback, make
    adjustments, and ensure user-friendliness.

    The platform goes beyond tool provision. It will offer a rich repository of literature, tutorials, and
    community-contributed content, aiding users in understanding our tools and textile processes. We aim to build a
    vibrant community, encouraging user contributions, collaboration, and innovation. Users can engage with our tools,
    participate in discussions, and contribute to open-source development.

    We'll provide extensive documentation, tutorials, and support to empower users to use our tools independently,
    regardless of their technical background. This platform establishes a foundation for HiDef Textiles' sustained
    growth. Our vision is for the community to drive innovation, enhance our tools, and share knowledge, ensuring the
    textile field remains vibrant and evolving.


    \section{Budget 2024-2026}

    Our budget for 2024-2026 aims to support the progression of the HiDef Textiles project, focusing on academic
    enrichment, technological development, and community engagement.

    Our initial plan includes providing a summer internship for a Master's student to engage in research and skill
    development.
    We are also allocating funds for potential licensing fees, ensuring the integration of advanced software tools
    like DALL·E and Midjourney for generative design.
    Once the foundational work is completed, we envision launching a beta version of our program on a web platform.
    This platform will include server-side applications to ensure functionality and user access.

    \subsection{Budget Breakdown}

    In order to offer a seamless and efficient user experience, we've identified Amazon Web Services as our
    hosting partner of choice for its robust and scalable infrastructure.
    Simplified breakdown of our proposed setup
    \begin{enumerate*}[itemjoin*={{, for }}]
        \item Amazon EC2 (t2.micro instance): \$102/year
        \item Amazon RDS (db.t2.micro instance): \$156/year
        \item Amazon S3 (moderate storage): \$60/year
        \item Data Transfer (moderate transfer): \$120/year
        \item Route 53 (domain + DNS): \$15/year
        \item ISNIC domain registration and annual renewal: 6,789 ISK/year
        \item grand total of approximately 50,000 ISK/year
    \end{enumerate*}

    The project will be open-source, and license fees will cover the use of generative AI software:
    \begin{enumerate*}[itemjoin*={{, for }}]
        \item Midjourney \$96 (standard plan) a year
        \item DALL·E operates on a \emph{pay-as-you-go} pricing model,
        \item Chat-GPT is \$300/year
        \item grand total for license fees of approximately 50,000 ISK/year
    \end{enumerate*}

    We anticipate an overhead of 100,000 ISK/year. Any remaining funds will be used to hire Master's students for 2-3
    months each summer from 2024-2026 to further develop the project.
    They are budgeted for 564,679 ISK/month (salary category no. 695.020) with hopes of hiring 1 student for 2-3
    months each year.

    If we encounter challenges in hiring suitable Master's students or face budget constraints, the remaining funds
    will be redirected to outsource the development of the project's graphical user interface and visual elements to
    a third party.

    \printbibliography

\end{document}