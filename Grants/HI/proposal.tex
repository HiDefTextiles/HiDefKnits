\documentclass{article}
\usepackage{geometry}
\geometry{a4paper, top=2.5cm, bottom=2.5cm, left=2.5cm, right=2.5cm}
% https://soloforms.lausn.is/input/?fid=783&rid=0fc742a8-14ec-4487-abdd-ff8731beb13b
\usepackage{changepage}
\usepackage{etoolbox}
\usepackage{setspace}
\onehalfspacing

\title{HiDef Textiles: Optimizing Textile Processes with AI}
\author{Dr. Helga Ingimundardóttir}
\date{\today}

\usepackage{natbib}
\bibliographystyle{plainnat}

\usepackage[inline]{enumitem}
\setlist[enumerate,1]{
    label=\textit{\roman*)},
    before=\unskip{: },
    itemjoin={{; }}, itemjoin*={{, and }},
    after={{. }}
}

\BeforeBeginEnvironment{abstract}{\begin{adjustwidth}{1cm}{1cm}\small\singlespacing}
\AfterEndEnvironment{abstract}{\end{adjustwidth}}

\begin{document}
    \maketitle
    %\begin{abstract}
    \textbf{HiDef Textiles} uniquely merges AI, industrial engineering, computer science, and textile craftsmanship,
    enhancing embroidery, sewing, knitwear design and more. Our vision is to democratize tech and programming via
    textiles, seamlessly integrating arts and sciences in an engaging manner.

    We are building a comprehensive learning platform, emphasizing sustainable practices and efficiency, and are
    committed to sharing knowledge and industrial best practices across all skill levels. Our educational
    integration fosters a collaborative space where arts, mathematics, and engineering converge.

    With a strong focus on encouraging women in STEM, we present textiles as a tangible entry point to
    engineering and computational concepts, aiming to cultivate an inclusive community and drive innovation.

    In seeking funding, our objective is to strengthen our STEAM approach, ensuring a rich, interdisciplinary
    experience that merges STEM with the Arts. This financial support will enable us to employ industrial
    engineering and computer science students for the development of both the content and the platform itself,
    thereby fostering a culture of knowledge and innovation.

    \noindent \textbf{Keywords:} \textit{Process Optimization}, \textit{Sustainable Textile Manufacturing},
    \textit{Interdisciplinary Innovation} and \textit{Generative Design}.
\end{abstract}

    \section{Introduction}

    Textiles, traditionally intertwined with women's work, embody a rich tapestry of skill, precision, and creativity.
    Ada Lovelace, a visionary in computing, recognized the algorithmic nature of the Jacquard loom, thereby linking
    textile arts to computation—a connection that, as \citet{sadieplant} emphasizes, has the potential to inspire women
    to delve into STEM fields, providing a hands-on way to understand engineering and computational concepts.

    HiDef Textiles stands at the intersection of traditional craftsmanship and modern technology, fostering a synergy
    between industrial engineering, computer science, and creative design. This collaborative initiative is not just a
    conduit for innovation but also a platform that champions diversity and inclusion, especially in encouraging
    women's participation in the field. Research works like those by \citet{qiu2013,keune2022} highlight the appeal
    of computational textiles, showcasing their effectiveness in making technology and programming less intimidating
    and more accessible.

    By intertwining artistic expression with engineering precision, HiDef Textiles is democratizing industrial
    processes, making them more accessible and inviting. In doing so, it contributes to the existing knowledge base,
    establishes innovative practices, and encourages diversity within the field, all while fostering a sense of
    community and shared learning.

    This initiative not only enhances the efficiency and sustainability of textile craftsmanship but also opens new
    avenues for learning and participation, ensuring a more inclusive future for the textile industry and beyond.


    \section{Projects}
    At HiDef Textiles, we intertwine the artistry of traditional textiles with the precision of modern technology and
    computational methods, striving for heightened efficiency, accessibility, and sustainability. Our work spans various
    facets of textile craftsmanship, and we highlight three pivotal initiatives below.

    \subsection{Optimization in Embroidery}

    Machine embroidery requires precision, and our initiative is aimed at boosting its
    efficiency. We are streamlining the workflow from design to stitching, employing
    operations research optimization techniques. Our goal is to significantly reduce
    setup and cleanup times, a major bottleneck in the embroidery process.

    In partnership with \citet{inkstitch}, an open-source embroidery design tool, we are
    working to integrate shortest route algorithms to enhance their toolkit. This task
    involves optimizing stitch points for efficiency and aesthetic appeal. Our efforts are
    streamlined to save time in both industrial setups and for hobbyists, ensuring
    precision and pleasing visual outcomes.

    \begin{description}
        \item[Objective:] Optimizing stitching paths used in machine embroidery.
        \item[Milestone 1:] Introduce basic concepts of shortest path problems to students.
        Students implement basic version for hands-on experience.
        \item[Milestone 2:] Add an additional layer of complexity to the problem by introducing an optimization for
        the start/end nodes in the sub-paths.
        \item[Milestone 3:] Develop advanced scenarios for master's projects focusing on processing paths for satin
        stitches, how they should meet and/or cross-over, and how to optimize the path for aesthetic appeal.
    \end{description}

    \subsection{Fabric Utilization in Sewing}

    Sewing requires precise pattern placement and fabric cutting, particularly in garment creation.
    This initiative aims to optimize these aspects using advanced algorithms. We address challenges
    like irregular fabric shapes and matching pattern repeats. Our goal is to minimize fabric waste,
    providing adaptable solutions for various fabric widths and suggesting ideal orientations for pattern pieces.

    We draw on methods from the 2D Free Form Bin Packing and 2D Irregular Cutting Stock Problems
    \citep{bennell2009tutorial, xu2016efficient}, targeting home sewists and small-scale garment
    makers. Our project helps these groups maximize fabric use, conserve resources, and adopt
    sustainable sewing practices. These users often work with non-uniform fabric shapes on a
    small scale, lacking access to advanced tools. Our open-source approach is thus invaluable.

    \begin{description}
        \item[Objective:] Optimize pattern placement and fabric utilization in garment creation by developing
        advanced algorithms to handle irregular fabric shapes and unique materials, to reduce fabric waste.

        \item[Milestone 1:] Begin with fundamental principles of fabric utilization, focusing on the
        optimal arrangement of regular shapes (rectangles) from a rectangular piece of fabric, i.e. cutting stock
        problem.

        \item[Milestone 2:] Transition to scenarios that involve regular shapes being cut from irregular
        pieces of fabric, teaching students how to maximize fabric usage with remnants or animal hides.

        \item[Milestone 3:] Introduce the complexity of cutting irregular shapes from regular fabric
        pieces, emphasizing efficient orientation and placement strategies. A common problem in industrial sewing.

        \item[Milestone 4:] Expand to real-world bespoke garment-making challenges, dealing with irregular
        shapes being cut from irregular fabric pieces.

        \item[Milestone 5:] Broaden the context to a 3D space, applying the same principles to
        3D printing, where students learn to place objects on a printer bed, navigating the added
        dimensions and complexities.
    \end{description}


    \subsection{AI-Enhanced Knitwear Design and Visualization}

    We are pioneering in AI-enhanced knitwear design and visualization, facilitating the adjustment of knitting
    patterns to individual needs, whether it be gauge discrepancies or yarn substitutions. This initiative benefits both
    knitters and designers, providing an easy means to adapt patterns and create versatile designs that accommodate
    varying gauges and yarn types.
    To fuel our innovation, we leverage extensive databases, including Ravelry\footnote{
        Ravelry, founded by \citeauthor{Ravelry} in \citeyear{Ravelry}, provides a comprehensive platform for fiber
        artists to share patterns, discuss techniques, and showcase finished projects. It enables users to browse and
        purchase patterns, track their own projects, and connect with a global community of enthusiasts. The platform
        hosts millions of user-generated patterns, making it an invaluable resource for understanding the knitting
        and crocheting community.
    }, a platform crucial for analyzing current trends, popular designs, and common pattern modifications.
    This deep understanding aids in the development of our generative AI applications, pushing the boundaries of
    pattern design and visualization.

    \begin{description}
        \item[Objective:] Develop generative AI applications for pattern instruction and modification of
        knitwear designs.
        \item[Milestone 1:] Analyze Ravelry dataset (meta-data) for trends and pattern popularity.
        \item[Milestone 2:] Develop generative AI for Fairisle patterns, using images of lopapeysa for training.
        \item[Milestone 3:] Identify and work on public domain patterns for future generative AI applications.
        \item[Milestone 4:] For complex knitwear like lace techniques, solicit participation from knitwear designers
        for pattern instruction sets.
        \item[Milestone 5:] Optimize patterns based on perceived quality, and reproduce using AI.
    \end{description}


    \section{Academic and Societal Value}

    HiDef Textiles stands at the intersection of innovation, education, and societal impact, with initiatives that
    range from major industrial applications to personal textile projects. Our commitment is to ensure wide-reaching
    relevance and tangible benefits across diverse domains.

    Integrating our projects into industrial engineering courses offers unique, hands-on learning experiences for
    students. We create opportunities for student engagement through smaller, practical projects, fostering a deep
    understanding of the real-world applications of their academic pursuits. Furthermore, we aim to attract master's
    students--hopefully funded--to make substantial contributions to our key initiatives, further intertwining
    academic growth with practical innovation.

    Our dedication to community empowerment shines through our open-source tools, designed to demystify textile
    processes and put control back into the users' hands. The HiDef Textiles web platform serves as a central hub for
    this exchange, enabling sharing, collaboration, and continuous learning within a vibrant community.

    Academic contributions and the dissemination of knowledge are central to our mission. We actively work to publish
    our findings in scholarly journals while ensuring that these complex concepts are translated into simplified
    learning resources. Our objective is to make knowledge accessible and understandable, broadening its impact and
    ensuring that it serves not just the academic community but also the wider public interested in textile arts. In
    doing so, HiDef Textiles bridges the gap between theory and practice, fostering a community of informed users,
    enthusiasts, and lifelong learners in the textile arts.

    \section{Conclusion}

    At HiDef Textiles, we are on a mission to transform textile manufacturing, striking a harmonious balance between
    efficiency, accessibility, and sustainability, all while preserving the invaluable heritage of traditional
    craftsmanship. Our initiatives pave the way for innovative learning opportunities, driving forward both academic
    excellence and societal well-being.

    By seamlessly integrating STEM disciplines into the realm of textile arts, we open doors to a diverse range of
    participants. We are especially keen on inspiring women to delve into the fields of optimization and operations
    research within industrial engineering, fostering gender diversity and inclusion in these critical areas of study.

    Our open-source platform stands at the core of our community-building efforts, acting as a catalyst for innovation,
    knowledge-sharing, and sustainable practices. It is a space where the textile community, and indeed the world at
    large, can come together to push the boundaries of what is possible, ensuring a future for textiles that is as
    rich and vibrant as its past.

    \newpage
    \bibliography{references}


\end{document}